\documentclass[hyperref]{ctexart}
\usepackage[left=2.50cm, right=2.50cm, top=1.50cm, bottom=2.50cm]{geometry} %页边距
\usepackage{helvet}
\usepackage{amsmath, amsfonts, amssymb} % 数学公式、符号
\usepackage[english]{babel}
\usepackage{graphicx}   % 图片
\usepackage{url}        % 超链接
\usepackage{bm}         % 加粗方程字体
\usepackage{multirow}
\usepackage{booktabs}
\usepackage{algorithm}
\usepackage{algorithmic}
\renewcommand{\algorithmicrequire}{ \textbf{Input:}}       
\renewcommand{\algorithmicensure}{ \textbf{Initialize:}} 
\renewcommand{\algorithmicreturn}{ \textbf{Output:}}     
%算法格式
\usepackage{fancyhdr} %设置页眉、页脚
\pagestyle{fancy}
\fancyhf{} 
\lhead{}
\chead{}
\lfoot{}
\cfoot{}
\rfoot{}
\renewcommand{\headrulewidth}{0} 
\renewcommand{\footrulewidth}{0} 
\usepackage{hyperref} %bookmarks
\hypersetup{colorlinks, bookmarks, unicode} %unicode
\usepackage{multicol}
\title{\textbf{不同劳资关系的激励效应对游戏公司的影响}}
\author{\sffamily 陈义桐 \\ \\复旦大学 \ 计算机科学与技术}
\date{(Dated: \today)}
\begin{document}
	\maketitle
		\noindent{\bf 摘要: }This is abstract.This is abstract.This is abstract.This is abstract.This is abstract.This is abstract.This is abstract.This is abstract.This is abstract.This is abstract.This is abstract.This is abstract.This is abstract.This is abstract.This is abstract.This is abstract.This is abstract.This is abstract.\\
		
		\noindent{\bf 关键词: }劳资关系;激励;隐马尔可夫模型;雅达利;暴雪娱乐;任天堂
	\begin{multicols}{2}
	\section{引言}
	This is introduction.This is introduction.This is introduction.This is introduction.This is introduction.This is introduction.This is introduction.This is introduction.This is introduction.This is introduction.This is introduction.
	\section{文献综述}
	This is introduction.This is introduction.This is introduction.This is introduction.This is introduction.This is introduction.
	\section{模型简介}
	本文采用隐马尔可夫模型进行建模,这里简单介绍一下该模型的建立过程。
	\subsection{模型原理}
	马尔科夫过程,是指当前状态只与前一个状态有关,而与更之前的状态无关的随机过程。\par
    即对于一个随机过程$\{X(t), t\in T\}$,对于任意取定的参数$t_1 < t_2 < ... < t_n$,
    \noindent 有: 
	\begin{equation*}
	\begin{aligned}
	    P(X(t_n) = x_n | X(t_i) = x_i, i \in \{1,...,n - 1\})\\ = 
		P(X(t_n) = x_n | X(t_{n-1}) = x_{n - 1})
	\end{aligned}
	\end{equation*}\par
	\indent\par 这样的一条状态序列称为马尔可夫链。\par
	隐马尔可夫模型(Hidden Markov Model)是一个统计模型,它用来描述一个含有两条马尔可夫链的随机过程,其中一条状态链可见,另一条状态链不可见。\par
    假设我们于一些骰子中盲选出一个骰子进行掷骰子操作,即无法观察具体每一次掷骰子的结果由哪一个骰子产生,则对于这样的一个多次随机过程,掷骰子的结果组成的序列便为可见状态链$o$,而每一次产生结果的骰子组成的序列便为隐含状态链$q$。\par
    对于隐马尔可夫过程,存在一些要素:\par 
    \noindent
    \begin{equation*}
    \begin{aligned}
        \pi &= (\pi_i):\pi_i = P(q_1 = i)\\
        A &= [a_{ij}]_{n \times m} :a_{ij} = P(q_{t+1} = j|q_t = i)\\
        B &= [b_j(k)]_{n \times m}: b_j(k) = P(o_t = o_k|q_t = j)
    \end{aligned}
    \end{equation*}\par
    分别为初始状态向量,即第一次拿到某个骰子的概率;状态转移概率矩阵,即假定现在拿到骰子$i$,下一个骰子是$j$的概率;观测概率矩阵,即假定拿到了骰子$j$,扔出结果$k$的概率。
    
    
    
	\section{title}
	\noindent Equations: 
	\begin{equation}
		E=mc^2
	\end{equation}
	\begin{equation}
		H\psi=E\psi
	\end{equation}\\
	\section{结论}
	This is conclusion. This is conclusion. This is conclusion. This is conclusion. This is conclusion. This is conclusion. This is conclusion. This is conclusion. This is conclusion.This is conclusion.
	
	\begin{thebibliography}{100}%此处数字为最多可添加的参考文献数量
		\bibitem{article1}This is reference.%title author journal data pages
		\bibitem{book1}This is reference.%title author publish date
	\end{thebibliography}
	\end{multicols}
\end{document}
