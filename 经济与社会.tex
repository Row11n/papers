\documentclass[hyperref]{ctexart}
\usepackage[left=2.3cm, right=2.3cm, top=1.50cm, bottom=2.50cm]{geometry} %页边距
\usepackage{helvet}
\usepackage{amsmath, amsfonts, amssymb} % 数学公式、符号
\usepackage[english]{babel}
\usepackage{graphicx}   % 图片
\usepackage{url}        % 超链接
\usepackage{bm}         % 加粗方程字体
\usepackage{multirow}
\usepackage{booktabs}
\usepackage{algorithm}
\usepackage{algorithmic}
\renewcommand{\algorithmicrequire}{ \textbf{Input:}}       
\renewcommand{\algorithmicensure}{ \textbf{Initialize:}} 
\renewcommand{\algorithmicreturn}{ \textbf{Output:}}     
%算法格式
\setlength\columnsep{0.77cm}
\usepackage{fancyhdr} %设置页眉、页脚
\pagestyle{plain}
\fancyhf{} 
\lhead{}
\chead{}
\lfoot{}
\cfoot{}
\rfoot{}
\renewcommand{\headrulewidth}{0} 
\renewcommand{\footrulewidth}{0} 
\usepackage{hyperref} %bookmarks
\hypersetup{colorlinks, bookmarks, unicode} %unicode
\usepackage{multicol}
\title{\textbf{不同控制权的分配方式对游戏公司的影响}}
\author{\sffamily 陈义桐 \\ \\复旦大学 \ 计算机科学与技术}
\date{(Dated: \today)}
\begin{document}
	\maketitle
		\noindent{\bf 摘要: }This is abstract.This is abstract.This is abstract.This is abstract.This is abstract.This is abstract.This is abstract.This is abstract.This is abstract.This is abstract.This is abstract.This is abstract.This is abstract.This is abstract.This is abstract.This is abstract.This is abstract.This is abstract.\\
		
		\noindent{\bf 关键词: }控制权分配;隐马尔可夫模型;雅达利;暴雪娱乐;任天堂
	\begin{multicols}{2}
	\section{引言}
	雅达利有限公司是上世纪70年代垄断级别的游戏产业公司。1983年,雅达利大崩溃,时任总裁Raymond Kassar原是一名没有任何开发电子游戏的技术背景经营纺织品的商人。2021年年末,暴雪娱乐公司权力高层被指控性别歧视、性骚扰,包括总裁在内的多位重要人事离职,他们大多是各个游戏项目的主要技术负责人与创意总监,这导致公司所有游戏项目停滞。游戏,其强烈的媒体交互性以及作品个人化作用极强导致的品牌黏性,使其不同于传统商业产品。在顶级游戏公司的运营中,传统的委托代理模式与互联网思维下的股份激励模式均受阻碍,如何合理进行控制权的分配成为顶级游戏公司需要解决的巨大难题。
	\section{模型简介}
	本文采用隐马尔可夫模型进行建模,这里简单介绍一下该模型的基本结构。
	\subsection{模型原理}
	马尔科夫过程,是指当前状态只与前一个状态有关,而与更之前的状态无关的随机过程。\par
    即对于一个随机过程$\{X(t), t\in T\}$,对于任意取定的参数$t_1 < t_2 < ... < t_n$,
    \noindent 有: \par
	\begin{equation*}
	\begin{aligned}
	    P(X(t_n) = x_n | X(t_i) = x_i, i \in \{1,...,n - 1\})\\ = 
		P(X(t_n) = x_n | X(t_{n-1}) = x_{n - 1})
	\end{aligned}
	\end{equation*}\par
	\indent\par 这样的一条状态序列称为马尔可夫链。\par
	隐马尔可夫模型(Hidden Markov Model)是一个统计模型,它用来描述一个含有两条马尔可夫链的随机过程,其中一条状态链可见,另一条状态链不可见。\par
    假设我们于一些骰子中盲选出一个骰子进行掷骰子操作,即无法观察具体每一次掷骰子的结果由哪一个骰子产生,则对于这样的一个多次随机过程,掷骰子的结果组成的序列便为可见状态链$o$,而每一次产生结果的骰子组成的序列便为隐含状态链$q$。\par
    对于隐马尔可夫模型,存在一些要素:\par \indent\par
    \noindent
    \begin{equation*}
    \begin{aligned}
        \pi &= (\pi_i):\pi_i = P(q_1 = i)\\
        A &= [a_{ij}]_{n \times m} :a_{ij} = P(q_{t+1} = j|q_t = i)\\
        B &= [b_j(k)]_{n \times m}: b_j(k) = P(o_t = o_k|q_t = j)
    \end{aligned}
    \end{equation*}\par \indent\par
    分别为初始状态向量,即第一次拿到某个骰子的概率;状态转移概率矩阵,即假定现在拿到骰子$i$,下一个骰子是$j$的概率;观测概率矩阵,即假定拿到了骰子$j$,扔出结果$k$的概率。\par
    通过确定要素的具体值,便可得到一个模型,此即隐马尔可夫模型。
    \subsection{模型定义}
    对于一家游戏公司,定义其生产游戏的成功指数$T$为可见状态变量,并设定成功阈值$T_{lim}$,当 $T > T_{lim}$ 时,我们认为该游戏是成功的;游戏成功与否的真值序列按时间顺序组成可见状态链$o$;定义其生产游戏的设计师的状态为隐含状态变量,与生产的游戏一一对应,组成隐含状态链$q$。进一步地,将所有设计师抽象为一位设计师,其状态用二元对$(\omega_i, \omega_r)$  表示,$\omega_i$  表示灵感值,$\omega_r$表示管理值。由于如何设计出成功的游戏作品已经超出本文的讨论范围,故我们假设该设计师设计的第一个游戏必然成功,即初始状态向量是满的$(\forall i, \pi_i = 1)$。由于讨论的是公司的策略问题,故假设每个公司的设计师相同,即观测概率矩阵相同。在3.1中,我们将基于雅达利、暴雪娱乐与任天堂三家游戏公司的数据进行状态转移概率矩阵的构造;在3.2中,我们将基于预测结果的匹配度,完善观测概率矩阵的构造,从而完成整个模型的建立。
    
    
	\section{模型构造}
	\subsection{状态转移概率矩阵的构造}
    1980年,雅达利的母公司时代华纳空降一位职业经理人Raymond Kassar,他是一个优秀的企业家,却没有电子游戏方面的技术背景。他要求开发人员夜以继日地工作,连最基本的休息时间都被压缩到最低限度,而开发者却不能从自己制作的游戏中获得半点好处,甚至都不能在包装上拥有自己的姓名。在这样的重压下,David Crane,Larry Kaplan,Alan Miller,Bob Whitehead四人离职雅达利,创立动视。动视认为“游戏开发者也是艺术家”,也乐于给开发者以艺术家的待遇。动视发行的游戏会在包装盒上印刷开发者的签名照和寄语,并且还会介绍游戏开发的背景故事和幕后花絮。这样的做法使得游戏具有了突出的个人属性,这样的“摇滚精神”从动视传出,深深地影响了整个游戏界,更是从动视继承至动视暴雪,继承至暴雪娱乐。雅达利与暴雪娱乐,这两家公司在控制权的分配上可以视作是相反的。
    \par
    任天堂创立于1889年,在1970年代后期投入电子游戏产业,几十年间一直不间断地推出高质量游戏。其灵魂人物“马里奥之父”宫本茂于1977年入职任天堂,四十多年间,他沉心制作游戏,从未进入任天堂权力高层,仅仅拥有少量管理权力。另一位任天堂的灵魂人物横井军平情况类似。任天堂是比较传统的委托代理结构,在任天堂刚刚进入电子游戏领域时,时任社长山内溥同样没有电子游戏技术背景。
    \par
    基于以上分析,我们首先将二元对$(\omega_i, \omega_r)$离散化,令$\omega_i , \omega_r \in \{1,2,3,4,5\} $,并且假设每个变量的可转移范围$|r| = 1$,即对于2,其可达状态为1,2,3;再假设其存在边缘效应,将0视作1,6视作5,即对于1,其可达范围为1,1, 2。故对于每个二元对$(\omega_i, \omega_r)$,其共有9个可转移状态变量,且共有25个不同二元对。
    \par
    不同公司对于员工的激励主要体现于$\omega_r$的转移,同时$\omega_r$的大小将会影响$\omega_i$的转移。我们认为雅达利对于员工的$\omega_r$呈现为低激励高风险的,即游戏若是成功的,$\omega_r$也很难向上迁移;游戏若是失败的,$\omega_r$则很容易向下迁移。暴雪的策略则与雅达利完全相反。对于任天堂,我们认为其是中激励中风险的,且$\omega_r$很难由4向5进行迁移。
    \par
    现在考虑$\omega_r$对于$\omega_i$的影响。显然,当$\omega_r$较低时,$\omega_i$几乎不受影响;当$\omega_r$较高时,$\omega_i$偏向于向下迁移。我们将较低与较高量化为$[1, 3]$与$[4,5]$。
    \par
    将上述分析进行简单的线性拟合,得到基础的状态转移概率矩阵$A = [a_{ij}]_{25 \times 9}$。由于需要考虑激励问题,即当目前游戏成功时,其$\omega_r$不会下降;反之,当目前游戏失败时,$\omega_r$不会上升,故还需要考虑矩阵的偏置。这里用一个简单的分摊函数\par
    \noindent
    \begin{equation}
        p_i^* = \begin{cases} 0, i \in J \\ \frac{p_i}{1 - \sum p_j}, i \notin J ,j \in J\end{cases}
    \end{equation}
    来进行偏置,其中$J$在游戏成功时表示为$\omega_r$下降概率集合,反之表示上升概率集合。
    \par\indent\par
    一个例子,假定(3,3)的状态转移概率:\\
    \\
    \begin{tabular}{c|ccccc}
      & (2,2) & (2,3)  & (2,4) & (3,2) & (3,3)\\
	(3,3) & 0.1 & 0.1& 0.1& 0.1& 0.2\\
	\hline & (3,4) & (4,2) & (4,3) & (4,4) \\
	(3,3) & 0.1& 0.1& 0.1& 0.1
    \end{tabular}\\
    \par
    当其游戏成功后,概率被偏置为:\\ \\
    \begin{tabular}{c|ccccc}
      & (2,2) & (2,3)  & (2,4) & (3,2) & (3,3)\\
	(3,3) & 0 & 1/7 & 1/7 & 0& 2/7\\
	\hline & (3,4) & (4,2) & (4,3) & (4,4) \\
	(3,3) & 1/7& 0& 1/7& 1/7
    \end{tabular}\\
    \par
    当其游戏失败后,概率被偏置为:\\ \\
    \begin{tabular}{c|ccccc}
      & (2,2) & (2,3)  & (2,4) & (3,2) & (3,3)\\
	(3,3) & 1/7 & 1/7 & 0 & 1/7& 2/7\\
	\hline & (3,4) & (4,2) & (4,3) & (4,4) \\
	(3,3) & 0& 1/7 & 1/7& 0
    \end{tabular}\\
    \par
    完整的状态转移概率矩阵数据已在附录中给出。
    \subsection{观测概率矩阵的构造}
	    本节综合国内外三家主流游戏测评媒体IGN、Gamespot与游民星空的分数,将各个公司游戏作品按时间顺序取平均值列出。由于雅达利时期较为久远,除民间口碑外,几乎没有评测数据,故主要以暴雪娱乐和任天堂为主。考虑疫情因素,采样的作品均于2020年以前正式发售。\\ \par
	   暴雪娱乐:\\\par
    \begin{tabular}{|c|c|}
    \hline
    魔兽世界 & 星际争霸II:自由之翼 & 9.33&9.23&
    \hline
    暗黑破坏神III&炉石传说&8.90& 8.57&
    \hline
    风暴英雄&守望先锋&8.4&9.17&
    \hline
    \end{tabular}\\ \\ \\ \par
    任天堂:\\ \\
     \begin{tabular}{|c|c|}
    \hline
    新超级马里奥兄弟U & 耀西的手工世界 & 8.87&8.23&
    \hline
    超级马里奥创作家2&马里奥卡丁车巡回赛界&8.90& 7.00&
    \hline
    路易吉洋馆3&马里奥与索尼克在东京奥运会&8.47&7.00&
    \hline
    \end{tabular}\\ \par
    由于在3.1的计算中,已经将$\omega_r$对于$\omega_i$的影响考虑其中,故此处将其视作两个独立的变量。且由于在3.1的计算中,$\omega_i$、$\omega_r$仅仅进行了简单的线性拟合,而游戏产品作为一种类艺术产品,存在明显的非线性特征,故此处考虑进行高斯拟合。显然地,$\omega_i$与成功指数$T$正相关,故可视作$\mu_i = 5$;即对于$\omega_i$,有
    \begin{equation}
        p_1(\omega_i) = \frac{1}{\sqrt{2\pi}\sigma_i}e^{-0.5(\frac{5 - \omega_i}{\sigma_i})^2}
    \end{equation}\par
    再考虑$\omega_r$。基于雅达利、暴雪与任天堂的事例分析,现在简单地认为,拥有一定的管理权有助于$T$, 但当$\omega_r$过高时,这样的帮助会锐减,故可视作$\mu_r = 4$,且$\sigma_r$较小;即对于$\omega_r$,有
    \begin{equation}
        p_2(\omega_r) = \frac{1}{\sqrt{2\pi}\sigma_r}e^{-0.5(\frac{4
        - \omega_i}{\sigma_i})^2}
    \end{equation}\par
    由于$T \in [1,10]$,且显然地,$\omega_i$对T的影响较$\omega_r$更高,故对于预测值$T*$,有
    \begin{equation}
        T* = 1.6\sum{\omega_i p^*_1(\omega_i)} + 0.4\sum {\omega_r p^*_2(\omega_r)}
    \end{equation}\par
    其中$p^*_1$、$p^*_2$表示利用(1)将定义域外的概率分摊进定义域内。\par
    基于(2)、(3)、(4),进行大量参数试算,得到当$\sigma_i = $,$\sigma_r = $时,基于暴雪娱乐的转移概率矩阵在给定初始状态为(x, x)时的产品值序列为 的概率为,基于任天堂的转移概率矩阵在给定初始状态为(x, x)时的产品值序列为 的概率为,与实际情况较为相符。\par
    故得到最终的观测概率映射函数:
	    
	\section{分析与反演}
	\subsection{分析}
	\subsection{反演}
	\section{结论}
	This is conclusion. This is conclusion. This is conclusion. This is conclusion. This is conclusion. This is conclusion. This is conclusion. This is conclusion. This is conclusion.This is conclusion.
	
	\begin{thebibliography}{100}%此处数字为最多可添加的参考文献数量
		\bibitem{article1}This is reference.%title author journal data pages
		\bibitem{book1}This is reference.%title author publish date
	\end{thebibliography}
	\end{multicols}
	
	
	\\ \\ \\ \\附录:\\ \par
	雅达利模型的状态转移概率矩阵:\\
	 \begin{tabular}{|c|ccccccccc|}
	 \hline
    (x,y)  & (x-1,y-1) & (x-1,y)  & (x-1,y+1) & (x,y-1) & (x,y)& (x,y+1)& (x+1,y-1)& (x+1,y) & (x+1,y+1)&
	\hline
	(1,1)  & 0.06 & 0.12  & 0.02 & 0.18 & 0.36 & 0.06& 0.06&0.12&0.02&
	(1,2)  & 0.06 & 0.12  & 0.02 & 0.18 & 0.36 & 0.06& 0.06&0.12&0.02& 
	(1,3)  & 0.06 & 0.12  & 0.02 & 0.18 & 0.36 & 0.06& 0.06&0.12&0.02&
	(1,4)  & 0.06 & 0.12  & 0.02 & 0.18 & 0.36 & 0.06& 0.06&0.12&0.02&
	(1,5)  & 0.06 & 0.12  & 0.02 & 0.18 & 0.36 & 0.06& 0.06&0.12&0.02& 
	(2,1)  & 0.06 & 0.12  & 0.02 & 0.18 & 0.36 & 0.06& 0.06&0.12&0.02&
	(2,2)  & 0.06 & 0.12  & 0.02 & 0.18 & 0.36 & 0.06& 0.06&0.12&0.02&
	(2,3)  & 0.06 & 0.12  & 0.05 & 0.18 & 0.36 & 0.04& 0.06&0.12&0.01&
	(2,4)  & 0.06 & 0.12  & 0.05 & 0.18 & 0.36 & 0.04& 0.06&0.12&0.01&
	(2,5)  & 0.06 & 0.12  & 0.05 & 0.18 & 0.36 & 0.04& 0.06&0.12&0.01&
	(3,1)  & 0.06 & 0.12  & 0.02 & 0.18 & 0.36 & 0.06& 0.06&0.12&0.02&
	(3,2)  & 0.06 & 0.12  & 0.02 & 0.18 & 0.36 & 0.06& 0.06&0.12&0.02&
	(3,3)  & 0.06 & 0.12  & 0.05 & 0.18 & 0.36 & 0.04& 0.06&0.12&0.01&
	(3,4)  & 0.06 & 0.12  & 0.05 & 0.18 & 0.36 & 0.04& 0.06&0.12&0.01&
	(3,5)  & 0.06 & 0.12  & 0.05 & 0.18 & 0.36 & 0.04& 0.06&0.12&0.01&
	(4,1)  & 0.06 & 0.12  & 0.02 & 0.18 & 0.36 & 0.06& 0.06&0.12&0.02&
	(4,2)  & 0.06 & 0.12  & 0.02 & 0.18 & 0.36 & 0.06& 0.06&0.12&0.02&
	(4,3)  & 0.06 & 0.12  & 0.065 & 0.18 & 0.36 & 0.03& 0.06&0.12&0.005&
	(4,4)  & 0.06 & 0.12  & 0.065 & 0.18 & 0.36 & 0.03& 0.06&0.12&0.005&
	(4,5)  & 0.06 & 0.12  & 0.065 & 0.18 & 0.36 & 0.03& 0.06&0.12&0.005&
	(5,1)  & 0.06 & 0.12  & 0.02 & 0.18 & 0.36 & 0.06& 0.06&0.12&0.02&
	(5,2)  & 0.06 & 0.12  & 0.02 & 0.18 & 0.36 & 0.06& 0.06&0.12&0.02&
	(5,3)  & 0.06 & 0.12  & 0.065 & 0.18 & 0.36 & 0.03& 0.06&0.12&0.005&
	(5,4)  & 0.06 & 0.12  & 0.065 & 0.18 & 0.36 & 0.03& 0.06&0.12&0.005&
	(5,5)  & 0.06 & 0.12  & 0.07 & 0.18 & 0.36 & 0.03& 0.06&0.12&0&
	\hline
    \end{tabular}\\ \\ \par
    
    暴雪娱乐模型的状态转移概率矩阵:\\
    \begin{tabular}{|c|ccccccccc|}
	 \hline
    (x,y)  & (x-1,y-1) & (x-1,y)  & (x-1,y+1) & (x,y-1) & (x,y)& (x,y+1)& (x+1,y-1)& (x+1,y) & (x+1,y+1)&
	\hline
	(1,1)  & 0.02 & 0.12  & 0.06 & 0.06 & 0.36 & 0.18& 0.02&0.12&0.06&
	(1,2)  & 0.02 & 0.12  & 0.06 & 0.06 & 0.36 & 0.18& 0.02&0.12&0.06&
	(1,3)  & 0.02 & 0.12  & 0.06 & 0.06 & 0.36 & 0.18& 0.02&0.12&0.06&
	(1,4)  & 0.02 & 0.12  & 0.06 & 0.06 & 0.36 & 0.18& 0.02&0.12&0.06&
	(1,5)  & 0.02 & 0.12  & 0.06 & 0.06 & 0.36 & 0.18& 0.02&0.12&0.06& 
	(2,1)  & 0.02 & 0.12  & 0.06 & 0.06 & 0.36 & 0.18& 0.02&0.12&0.06&
	(2,2)  & 0.02 & 0.12  & 0.06 & 0.06 & 0.36 & 0.18& 0.02&0.12&0.06&
	(2,3)  & 0.02 & 0.12  & 0.11 & 0.06 & 0.36 & 0.15& 0.02&0.12&0.04&
	(2,4)  & 0.02 & 0.12  & 0.11 & 0.06 & 0.36 & 0.15& 0.02&0.12&0.04&
	(2,5)  & 0.02 & 0.12  & 0.11 & 0.06 & 0.36 & 0.15& 0.02&0.12&0.04&
	(3,1)  & 0.02 & 0.12  & 0.06 & 0.06 & 0.36 & 0.18& 0.02&0.12&0.06&
	(3,2)  & 0.02 & 0.12  & 0.06 & 0.06 & 0.36 & 0.18& 0.02&0.12&0.06&
	(3,3)  & 0.02 & 0.12  & 0.11 & 0.06 & 0.36 & 0.15& 0.02&0.12&0.04&
	(3,4)  & 0.02 & 0.12  & 0.11 & 0.06 & 0.36 & 0.15& 0.02&0.12&0.04&
	(3,5)  & 0.02 & 0.12  & 0.11 & 0.06 & 0.36 & 0.15& 0.02&0.12&0.04&
	(4,1)  & 0.02 & 0.12  & 0.06 & 0.06 & 0.36 & 0.18& 0.02&0.12&0.06&
	(4,2)  & 0.02 & 0.12  & 0.06 & 0.06 & 0.36 & 0.18& 0.02&0.12&0.06&
	(4,3)  & 0.02 & 0.12  & 0.16 & 0.06 & 0.36 & 0.12& 0.02&0.12&0.02&
	(4,4)  & 0.02 & 0.12  & 0.16 & 0.06 & 0.36 & 0.12& 0.02&0.12&0.02&
	(4,5)  & 0.02 & 0.12  & 0.16 & 0.06 & 0.36 & 0.12& 0.02&0.12&0.02&
	(5,1)  & 0.02 & 0.12  & 0.06 & 0.06 & 0.36 & 0.18& 0.02&0.12&0.06&
	(5,2)  & 0.02 & 0.12  & 0.06 & 0.06 & 0.36 & 0.18& 0.02&0.12&0.06&
	(5,3)  & 0.02 & 0.12  & 0.18 & 0.06 & 0.36 & 0.11& 0.02&0.12&0.01&
	(5,4)  & 0.02 & 0.12  & 0.18 & 0.06 & 0.36 & 0.11& 0.02&0.12&0.01&
	(5,5)  & 0.02 & 0.12  & 0.19 & 0.06 & 0.36 & 0.11& 0.02&0.12&0&
	\hline
    \end{tabular}\\ \\ \par
    
    任天堂模型的状态转移概率矩阵:\\
    \begin{tabular}{|c|ccccccccc|}
	 \hline
    (x,y)  & (x-1,y-1) & (x-1,y)  & (x-1,y+1) & (x,y-1) & (x,y)& (x,y+1)& (x+1,y-1)& (x+1,y) & (x+1,y+1)&
	\hline
	(1,1)  & 0.01 & 0.16  & 0.03 & 0.03 & 0.48 & 0.09& 0.01&0.16&0.03&
	(1,2)  & 0.01 & 0.16  & 0.03 & 0.03 & 0.48 & 0.09& 0.01&0.16&0.03&
	(1,3)  & 0.01 & 0.16  & 0.03 & 0.03 & 0.48 & 0.09& 0.01&0.16&0.03&
	(1,4)  & 0.01 & 0.16  & 0.03 & 0.03 & 0.48 & 0.09& 0.01&0.16&0.03&
	(1,5)  & 0.01 & 0.16  & 0.03 & 0.03 & 0.48 & 0.09& 0.01&0.16&0.03& 
	(2,1)  & 0.01 & 0.16  & 0.03 & 0.03 & 0.48 & 0.09& 0.01&0.16&0.03&
	(2,2)  & 0.01 & 0.16  & 0.03 & 0.03 & 0.48 & 0.09& 0.01&0.16&0.03&
	(2,3)  & 0.01 & 0.16  & 0.05 & 0.03 & 0.48 & 0.08& 0.01&0.16&0.02&
	(2,4)  & 0.01 & 0.16  & 0.05 & 0.03 & 0.48 & 0.08& 0.01&0.16&0.02&
	(2,5)  & 0.01 & 0.16  & 0.05 & 0.03 & 0.48 & 0.08& 0.01&0.16&0.02&
	(3,1)  & 0.01 & 0.16  & 0.03 & 0.03 & 0.48 & 0.09& 0.01&0.16&0.03&
	(3,2)  & 0.01 & 0.16  & 0.03 & 0.03 & 0.48 & 0.09& 0.01&0.16&0.03&
	(3,3)  & 0.01 & 0.16  & 0.05 & 0.03 & 0.48 & 0.08& 0.01&0.16&0.02&
	(3,4)  & 0.01 & 0.16  & 0.05 & 0.03 & 0.48 & 0.08& 0.01&0.16&0.02&
	(3,5)  & 0.01 & 0.16  & 0.05 & 0.03 & 0.48 & 0.08& 0.01&0.16&0.02&
	(4,1)  & 0.01 & 0.16  & 0.03 & 0.03 & 0.48 & 0.09& 0.01&0.16&0.03&
	(4,2)  & 0.01 & 0.16  & 0.03 & 0.03 & 0.48 & 0.09& 0.01&0.16&0.03&
	(4,3)  & 0.01 & 0.16  & 0.075 & 0.03 & 0.48 & 0.06& 0.01&0.16&0.015&
	(4,4)  & 0.01 & 0.16  & 0.075 & 0.03 & 0.48 & 0.06& 0.01&0.16&0.015&
	(4,5)  & 0.01 & 0.16  & 0.075 & 0.03 & 0.48 & 0.06& 0.01&0.16&0.015&
	(5,1)  & 0.01 & 0.16  & 0.03 & 0.03 & 0.48 & 0.09& 0.01&0.16&0.03&
	(5,2)  & 0.01 & 0.16  & 0.03 & 0.03 & 0.48 & 0.09& 0.01&0.16&0.03&
	(5,3)  & 0.01 & 0.16  & 0.08 & 0.03 & 0.48 & 0.06& 0.01&0.16&0.01&
	(5,4)  & 0.01 & 0.16  & 0.08 & 0.03 & 0.48 & 0.06& 0.01&0.16&0.01&
	(5,5)  & 0.01 & 0.16  & 0.09 & 0.03 & 0.48 & 0.06& 0.01&0.16&0&
	\hline
    \end{tabular}\\ \\ \par
	
\end{document}
